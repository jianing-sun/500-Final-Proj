\documentclass [12pt,letterpaper]{exam}
\usepackage{amsmath, amsthm, amsfonts, amssymb, amscd, latexsym}
\usepackage{type1cm}
\usepackage{simplemath}
\def\be{\begin{equation}}
\def\ee{\end{equation}}
\def\bea{\begin{eqnarray}}
\def\eea{\end{eqnarray}}
\usepackage{indentfirst}
\oddsidemargin  0.0in \evensidemargin 0.0in \textwidth      6.0in
\headheight     0.0in \topmargin      1.0in \textheight     9.0in

\header{ECSE-500-01-1}{Assignment 7}{17.10.2005}

\newcounter{count}

%%%%%%%%%%%%%%%%%%%%%%%%%%%%%%%%%%%%%%%%%%%%%%%%%%%%%%%%%%%%%%%%%%%%%%%%%%%%%%%%%%%%%%%%%%%%%%%%%%%%

\begin{document}

\begin{questions}

\question Show that if $s_n \leq t_n$ for all $n \geq N$, where $N$
is some fixed integer, then \be \liminf_{n \rightarrow \infty} s_n
\leq \liminf_{n \rightarrow \infty} t_n \nonumber \ee \be \limsup_{n
\rightarrow \infty} s_n \leq \limsup_{n \rightarrow \infty} t_n
\nonumber \ee\\
Proof: Since $s_n \leq t_n$ for all $n \geq N$, when $n \rightarrow \infty$, $n$ must $\geq $ the fixed integer N. Then $s_n \leq t_n$. $\lim_{n\to\infty}$inf $s_n$ is the lowest limit of sequence $s_n$, hence:$$\liminf_{n \rightarrow \infty} s_n \leq \liminf_{n \rightarrow \infty}t_n$$
\par \setlength
\indent Similarly, we know that $\lim_{n\to\infty}$sup $s_n$ is the greatest limit of sequence $s_n$, the upper limit $s^*$ and $t^*$ belong to $s_n$, and $s_n \leq t_n$ for all $n \geq N$, therefore:$$\limsup_{n \rightarrow \infty} s_n \leq \limsup_{n \rightarrow \infty}$$\\
\setlength{\parindent}{1cm}

\question Find the upper and lower limits of the sequence $\{ s_n
\}$ defined by: \be s_1 =0,\hspace{5mm}
s_{2m}=\frac{s_{2m-1}}{2},\hspace{5mm} s_{2m+1}=\frac{1}{2}+s_{2m}
\nonumber \ee

Proof: We show that $$(s_{2n-1}, s_{2n}) = (\frac{2^{n-1} - 1}{2^n-1},\frac{2^{n-1}-1}{2^n}), n \geq 1.$$
In fact, by the definition, $$(s_1,s_2)=(0,\frac{0}{2})=(0,0).$$
Suppose the formula holds for $n = k$. Then
\begin{equation} 
\begin{split}
(s_{2(k+1)-1},s_{2(k+1)}) & = (\frac{1}{2}+s_{2k},\frac{s_{2k+1}}{2}) = (\frac{1}{2}+\frac{2^{k-1}-1}{2^k},\frac{\frac{1}{2}+s_{2k}}{2})\\
 & = (\frac{2^{k+1}-2}{2^{k+1}},\frac{1}{4},\frac{1}{2}\frac{2^{k-1}-1}{2^k})\\
 & = (\frac{2^{(k+1)-1}-2}{2^{(k+1)-1}},\frac{2^{(k+1)-1}}{2^{k+1}})
\end{split}
\end{equation}
The proved expression for ${s_n}$ gives:
$$\lim_{n \rightarrow \infty}s_{2n-1} = 1, \lim_{n \rightarrow \infty}s_{2n-1} = \frac{1}{2}.$$
Hence, by the definitions of upper and lower limits, we have
$$\limsup_{n \rightarrow \infty}s_n = \overline{{\lim_{n \rightarrow \infty}}}s_n = 1, \liminf_{n \rightarrow \infty} = \underline{{\lim_{n \rightarrow \infty}}}s_n = \frac{1}{2}$$\\

\question Suppose $\{p_n \}$ is a Cauchy sequence in a metric space
$X$, and some subsequences $\{p_{n_i}\}$ converges to a point $p \in
X$. Prove that the full sequence $\{p_n \}$ converges to $p$.\\\\
Proof: Since ${p_n}$ is a Cauchy sequence and ${p_{n_i}}$ converges to a point $p \in X$, given $\varepsilon > 0$ there exists $N_1,N_2$ such that
$$d(p_n,p_m) < \frac{\varepsilon}{2}, n,m \geq N_1$$
$$d(p_{n_i},p) < \frac{\varepsilon}{2}, i \geq N_2$$
Assume that $n_i \leq n_{i+1}$, hence
$$d(p_n,p) \leq d(p_n,p_{n_i}) + d(p_{n_i},p) < \frac{\varepsilon}{2} + \frac{\varepsilon}{2} = \varepsilon$$
if $n,n_i \geq N = max(N-1,n_{N_2})$, since $\varepsilon > 0$ is arbitrary, $p_n \rightarrow p$\\


\question Suppose $\{p_n \}$ and $\{q_n \}$ are Cauchy sequences in
a metric space $X$. Show that the sequence $\{d(p_n,q_n)\}$
converges.\\[3mm]
{\em Hint:} For any $m,n$ \be d(p_n,q_n) \leq
d(p_n,p_m)+d(p_m,q_m)+d(q_m,q_n) \nonumber \ee\\
Proof: For any $m,n, d(p_n,q_n) \leq d(p_n,p_m)+d(p_m,q_m)+d(q_m,q_n)$, it follows that $|d(p_n,q_n) - d(p_m,q_m)$ is small if $m$ and $n$ are large.\\
For any $\varepsilon > 0$, there exits $N$ such that $d(p_n,p_m) < \varepsilon$ and $d(q_m,q_n) < \varepsilon$ whenever $m,n \geq N$. Note that $$d(p_n,q_n) \leq d(p_n,p_m) + d(p_m,q_m) + d(q_m,q_n)$$
It allows that $$|d(p_n,q_n) - d(p_m,q_m)| \leq d(p_n,p_m) + d(q_m,q_n) < 2\varepsilon$$
Thus ${d(p_n,q_n)}$ is a Cauchy sequence in $X$. Hence ${d(p_n,q_n)}$ converges.\\

\question If $f$ is a continuous mapping of a metric space $X$ into
a metric space $Y$, prove that \be f(\overline{E}) \subset
\overline{f(E)} \nonumber \ee for every set $E \subset X$.
($\overline{E}$ denotes the closure of $E$).\\
Proof: $\overline{f(E)}$ is a closed subset of $Y$ and $f$ is continuous, so $f^{-1}(\overline{f(E)})$ is a closed subset of $X$. Clearly $E \subset f^{-1}(\overline{f(E)})$, and the latter closed, so $\overline{E} \subset f^{-1}(\overline{f(E)})$, or $f(\overline{E}) \subset \overline{f(E)}$.\\



\question Let $f$ and $g$ be continuous mappings of a metric space
$X$ into a metric space $Y$, and let $E$ be a dense subset of $X$.
Prove that $f(E)$ is dense in $f(X)$. If $g(p)=f(p)$ for all $p \in
E$, prove that $g(p)=f(p)$ for all $p \in X$.\\\\
Proof: To check that $f(E)$ is dense in $f(X)$, it suffices to check that for every open subset $U \subset Y$ such that $U \cap f(X) \neq \emptyset$, $f(E) \cap (U \cap f(X)) \neq \emptyset$. \\
But $f^{-1}(U \cap f(X)) = f^{-1}(U)$ is then nonempty and is open in $X$ because $f$ is continuous. E is dense in $X$, so $E \cap f^{-1}(U) \neq \emptyset$.\\
Therefore if $x \in E \cap f^{-1}(U)$ then $f(x) \in f(E) \cap U = f(E) \cap (U \cap f(X))$, so the latter set is nonempty.\\
Let $S = {x \in X | f(x) = g(x)}$. Firstly, $S$ is closed by checking its complement $S^c = {x \in X | f(x) \neq g(x)}$ is an open subset of $X$. If $x \in X$ is such that $f(x) \neq g(x), i.e. x \in S^c$, then let $r = d_Y(f(x),g(x)) > 0$.\\
Then $B_{r/2}(f(x)) \cap B_{r/2}(g(x)) = \emptyset$ in $Y$, because if $z \in B_{r/2}(f(x))$ then $r = d(f(x),g(x)) \leq d(f(x),z) + d(z,g(x))$, or $d(g(x),z) \geq d(f(x),g(x)) - d(f(x),z) = r - d(f(x),z) > r-\frac{r}{2} = \frac{r}{2}$, hence $z \notin B_{r/2}(g(x))$.\\
Since $f,g$ are both continuous, $V = f^{-1}(B_{r/2}(g(x)))$ is an open subset of $X$ containing $x$ which has the property that if $y \in V$ then $f(y) \in B_{r/2}(g(x))$ and $g(y) \in B_{r/2}(g(x))$ so that $f(y) \neq g(y)$ since $B_{r/2}(f(x)) \cap B_{r/2}(g(x)) = \emptyset$.\\
Therefore $x \in V \subset S^c$, so $S^c$ is open. \\
Therefore $S$ is closed.<<
Now that $E \subset S$, and $S$ is a closed subset of $X$, so $X = \overline{E} \subset \overline{S} = S \subset X$.\\
Therefore $S = X$, and $f(x) = g(x)$ for all $x \in X$.\\


\question Let $f$ be a real, uniformly continuous function on a
bounded set $E$ in $R^1$. Prove that $f$ is bounded on $E$. Show
that the conclusion is false if boundedness of $E$ is omitted from
the hypothesis.\\\\
Proof: Since $E$ is a bounded set in $\mathbb{R}$, there is some $R > 0$ such that $E \subset [-R,R].$ Set $\varepsilon = 1$. Since $f$ is uniformly continuous, there is some $\delta > 0$ such that $|x - y| < \delta$ implies that $|f(x) - f(y)| < \varepsilon = 1$. Let $n \in \mathbb{N}$ be the smallest integer such that ${\frac{1}{n} < \delta}$, and let $$[-R,-R+\frac{1}{n}], [-R+\frac{1}{n},-R+\frac{2}{n}],...,[R-\frac{2}{n},R-\frac{1}{n}],[R-\frac{1}{n},R]$$
be a collection $S$ of $2nR$ invervals that exactly line up and cover $[-R,R]$. \\
By omitting some if necessary, let $I_1, I_2,..., I_N$ be those intervals from the collection that overlap with $E$, i.e. such that $I_k \cap E \Rightarrow |x_k - x| < \frac{1}{n} < \delta \Rightarrow |f(x)| < 1 + |f(x_k)|$ \\
Letting $M = max_{1 \leq k \leq N}(1 + f(x_k))$, for any $x \in E$ we have $|f(x)| < M$, so $f$ is a bounded function on $E$.\\
The second part asks us to show the conclusion is false if boundedness of $E$ is omitted from the hypothesis. Let $E = \mathbb{R}$, which is unbounded, and let $f(x)$, which is uniformly continuous because for any $\varepsilon > 0$, let $\delta = \varepsilon$ and we have
$|x - y| < \delta \Rightarrow |f(x) - f(y)| = |x - y| < \delta = \varepsilon$ \\
However, $f(x) = x$ is not bounded on $E = \mathbb{R}$\\

\question If $f$ is defined on $E$, the {\em graph} of $f$ is the
set of points $(x,f(x))$, for $x \in E$. In particular, if $E$ is a
set of real numbers, and $f$ is real-valued, the graph of $f$ is a
subset of the plane. Suppose $E$ is compact, and prove that $f$ is
continuous on $E$ if and only if its graph is compact.
\\\\
Proof: ($\Rightarrow$) Let $G = {(x,f(x)):x \in E}$. Since $f$ is a continuous mapping of a compact set $E$ into $f(E)$, from Theorem 4.14 in Lecture 6: A mapping $f$ of a set $E$ into $R^k$ is said to be bounded if there is a real number $M$ such that $|f(x)| \leq M$ for all $x \in E$.\\
Hence $f$ is also compact. Suppose the product of finitely many compact sets is compact. Thus $G = E \times f(E)$ is also compact.\\
($\Leftarrow$) Define $$g(x) = (x,f(x))$$\\
from $E$ to $G$ for $x \in E$. Suppose that $g(x)$ is continuous on $E$.Consider $h(x,f(x)) = x$ from $G$ to $E$. Hence $h$ is injective, continuous on a compact set $G$. \\
Therefore its inverse function $g(x)$ is injective and continuous on a compact set $E$. Since $g(x)$ is continuous on $E$, the component of $g(x)$ is continuous on a compact set $E$. That means $f(x)$ is continuous on a compact set $E$.\\
 
\question Let $f$ be a continuous real function on a metric space
$X$. Let $Z(f)$ (the {\em zero set} or {\em kernel} of $f$) be the
set of all $p \in X$ at which $f(p)=0$. Prove that $Z(f)$ is closed.\\
Proof: Since $Z(f) = f^{-1}({0})$, and ${0}$ is a closed set in $\mathbb{R}$, by the Corollary of Theorem 4.8 from lecture 6: A mapping $f$ of a metric space $X$ into a metric space $Y$ is continuous if and only if $f^{-1}(C)$ is closed in $X$ for every closed set $C$ in $Y$.\\
Hence $Z(f)$ is closed if $f$ is continuous.\\
Let $p$ be a limit point of $Z(f)$. Then there exists a sequence ${p_n}$ in $Z(f)$ such that $d(p_n,o) \rightarrow 0$. Since $f$ is continuous at $p$, for any $\varepsilon > 0$, there exists $\delta > 0$ such that $d(x,p) < \delta$ implies $|f(x) - f(p)| < \varepsilon$.\\
By $d(p_n,p) \rightarrow 0$, we know that there exists $N$ such that $n \geq N$ implies $d(p_n,p) < \delta$. 
Hence, if $n \geq N$, then $|f(p_n) - f(p)| < \varepsilon$. Since $f(p_n) = 0$, we know that $|f(p)| < \varepsilon$ for any $\varepsilon > 0$. This implies $f(p) = 0$, or $p \in Z(f)$. Therefore $Z(f)$ is closed since it contains all its limit points.\\

\end{questions}
\end{document}
